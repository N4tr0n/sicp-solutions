% Created 2019-01-01 Tue 20:23
% Intended LaTeX compiler: pdflatex
\documentclass[11pt]{article}
\usepackage[utf8]{inputenc}
\usepackage[T1]{fontenc}
\usepackage{graphicx}
\usepackage{grffile}
\usepackage{longtable}
\usepackage{wrapfig}
\usepackage{rotating}
\usepackage[normalem]{ulem}
\usepackage{amsmath}
\usepackage{textcomp}
\usepackage{amssymb}
\usepackage{capt-of}
\usepackage{hyperref}
\author{Nathan Hitchings}
\date{\today}
\title{Assignment 2 Solutions}
\hypersetup{
 pdfauthor={Nathan Hitchings},
 pdftitle={Assignment 2 Solutions},
 pdfkeywords={},
 pdfsubject={},
 pdfcreator={Emacs 26.1 (Org mode 9.1.14)},
 pdflang={English}}
\begin{document}

\maketitle
\section{Exercise 1.9}
\label{sec:org37831e6}
Each of the following two procedures defines a method for adding two positive
integers in terms of the procedures \texttt{inc}, which increments its argument by 1,
and \texttt{dec}, which decrements its argument by 1.
\begin{verbatim}
(define (+ a b)
  (if (= a 0)
      b
      (inc (+ (dec a) b))))

(define (+ a b)
  (if (= a 0)
      b
      (+ (dec a) (inc b))))
\end{verbatim}
\subsection{Solution}
\label{sec:orgfd8ab2a}
\begin{verbatim}
(inc (+ (dec 4) 5))
(inc (+ 3 5))
(inc (inc (+ (dec 3) 5)))
(inc (inc (+ 2 5)))
(inc (inc (inc (+ (dec 2) 5))))
(inc (inc (inc (+ 1 5))))
(inc (inc (inc (inc (+ (dec 1) 5)))))
(inc (inc (inc (inc (+ 0 5)))))
(inc (inc (inc (inc 5))))
(inc (inc (inc 6)))
(inc (inc 7))
(inc 8)
9

(+ (dec 4) (inc 5))
(+ 3 6)
(+ (dec 3) (inc 6))
(+ 2 7)
(+ (dec 2) (inc 7))
(+ 1 8)
(+ (dec 1) (inc 8))
(+ 0 9)
9
\end{verbatim}
The first process is recursive while the second process is iterative.

\section{Exercise 1.10}
\label{sec:org9ff7ee2}
The following procedure computes a mathematical function called Ackermann's
function.
\begin{verbatim}
(define (A x y)
  (cond ((= y 0) 0)
        ((= x 0) (* 2 y))
        ((= y 1) 2)
        (else (A (- x 1)
                 (A x (- y 1))))))
\end{verbatim}

What are the values of the following expressions?
\begin{verbatim}
(A 1 10)
(A 2 4)
(A 3 3)
\end{verbatim}

Consider the following procedures, where \texttt{A} is the procedure defined above:
\begin{verbatim}
(define (f n) (A 0 n))
(define (g n) (A 1 n))
(define (h n) (A 2 n))
(define (k n) (* 5 n n))
\end{verbatim}

Give concise mathematical definitions for the functions computed by the
procedures \texttt{f}, \texttt{g}, and \texttt{h} for positive integer values of \texttt{n}. For example,
\texttt{(k n)} computes \(5n^2\)

\subsection{Solution}
\label{sec:org77f5864}
\begin{verbatim}
(A 1 10)
(A 0 (A 1 9))
(A 0 (A 0 (A 1 8)))
(A 0 (A 0 (A 0 (A 1 7))))
(A 0 (A 0 (A 0 (A 0 (A 1 6)))))
(A 0 (A 0 (A 0 (A 0 (A 0 (A 1 5))))))
(A 0 (A 0 (A 0 (A 0 (A 0 (A 0 (A 1 4)))))))
(A 0 (A 0 (A 0 (A 0 (A 0 (A 0 (A 0 (A 1 3))))))))
(A 0 (A 0 (A 0 (A 0 (A 0 (A 0 (A 0 (A 0 (A 1 2)))))))))
(A 0 (A 0 (A 0 (A 0 (A 0 (A 0 (A 0 (A 0 (A 0 (A 1 1))))))))))
(A 0 (A 0 (A 0 (A 0 (A 0 (A 0 (A 0 (A 0 (A 0 2)))))))))
(A 0 (A 0 (A 0 (A 0 (A 0 (A 0 (A 0 (A 0 4))))))))
(A 0 (A 0 (A 0 (A 0 (A 0 (A 0 (A 0 8)))))))
(A 0 (A 0 (A 0 (A 0 (A 0 (A 0 16))))))
(A 0 (A 0 (A 0 (A 0 (A 0 32)))))
(A 0 (A 0 (A 0 (A 0 64))))
(A 0 (A 0 (A 0 128)))
(A 0 (A 0 256))
(A 0 512)
1024

(A 2 4)
(A 1 (A 2 3))
(A 1 (A 1 (A 2 2)))
(A 1 (A 1 (A 1 (A 2 1))))
(A 1 (A 1 (A 1 2)))
(A 1 (A 1 (A 0 (A 1 1))))
(A 1 (A 1 (A 0 2)))
(A 1 (A 1 4))
(A 1 (A 0 (A 1 3)))
(A 1 (A 0 (A 0 (A 1 2))))
(A 1 (A 0 (A 0 (A 0 (A 1 1)))))
(A 1 (A 0 (A 0 (A 0 2))))
(A 1 (A 0 (A 0 4)))
(A 1 (A 0 8))
(A 1 16)
(A 0 (A 1 15))
(A 0 (A 0 (A 1 14)))
(A 0 (A 0 (A 0 (A 1 13))))
(A 0 (A 0 (A 0 (A 0 (A 1 12)))))
(A 0 (A 0 (A 0 (A 0 (A 0 (A 1 11))))))
(A 0 (A 0 (A 0 (A 0 (A 0 (A 0 (A 1 10)))))))
(A 0 (A 0 (A 0 (A 0 (A 0 (A 0 1024))))))
(A 0 (A 0 (A 0 (A 0 (A 0 (A 0 1024))))))
(A 0 (A 0 (A 0 (A 0 (A 0 2048)))))
(A 0 (A 0 (A 0 (A 0 4096))))
(A 0 (A 0 (A 0 8192)))
(A 0 (A 0 16384))
(A 0 32768)
65536

(A 3 3)
(A 2 (A 3 2))
(A 2 (A 2 (A 3 1)))
(A 2 (A 2 2))
(A 2 (A 1 (A 2 1)))
(A 2 (A 1 2))
(A 2 4)
65536
\end{verbatim}

\texttt{(f n)} computes \(2n\).
\texttt{(g n)} computes \(2^n\).
\texttt{(h n)} computes \(^n2\).

\section{Exercise 1.11}
\label{sec:orgdd8356f}
A function \emph{f} is defined by the rule that
\begin{equation}
f(n)=
\begin{cases}
n & n<3 \\ f(n-1) + 2f(n-2) + 3f(n-3) & n\geq 3
\end{cases}
\end{equation}
Write a procedure that computes \emph{f} by means of a recursive procedure. Write a
procedure that computes \emph{f} by means of an iterative procedure

\subsection{Solution}
\label{sec:org8f466a0}
\begin{verbatim}
(define (f-recur n)
  (if (< n 3)
      n
      (+ (f-recur (- n 1))
         (* 2 (f-recur (- n 2)))
         (* 3 (f-recur (- n 3))))))

(define (f n)
  (define (iter count a b c)
    (if (>= count n)
        a
        (iter (+ count 1)
              b
              c
              (+ (* 3 a)
                 (* 2 b)
                 c))))
  (iter 0 0 1 2))
\end{verbatim}

\section{Exercise 1.12}
\label{sec:org4819307}
The following pattern of numbers is called \emph{Pascal's triangle}.

\begin{verbatim}
1
1 1
1 2 1
1 3 3 1
1 4 6 4 1
   ...
\end{verbatim}

The numbers at the edge of the triangle are all 1, and each number inside the
triangle is the sum of the two numbers above it. Write a procedure that
computes elements of Pascal's triangle by means of a recursive process.
\subsection{Solution}
\label{sec:org5e3d6db}
\begin{verbatim}
(define (pascal row column)
  (cond ((or (< column 0)
             (> column row))
         0)
        ((or (= row 0)
             (= column 0)
             (= row column))
         1)
        (else
         (+ (pascal (- row 1)
                    (- column 1))
            (pascal (- row 1)
                    column)))))
\end{verbatim}

\section{Exercise 1.13}
\label{sec:org6aa625d}
Prove that \(\text{Fib}(n)\) is the closest integer to \(\phi^n/\sqrt{5}\), where \(\phi = (1+\sqrt{5})/2\). Hint: Let \(\psi = (1-\sqrt{5})/2\). Use induction and the definition of the Fibonacci numbers to prove that \(\text{Fib}(n) = (\phi^n - \psi^n)/\sqrt{5}\).

\subsection{Solution}
\label{sec:org4d41060}
This solution is incomplete.
The formula for the nth fibonacci number is
\begin{equation*}
\text{Fib}(n)=
\begin{cases}
0 & \text{if } n = 0 \\
1 & \text{if } n = 1 \\
\text{Fib}(n-1)+\text{Fib}(n-2) & \text{otherwise}
\end{cases}
\end{equation*}
Setting \(n=0\), we have \(\frac{\phi^0}{\sqrt{5}} =
   \frac{1}{\sqrt{5}}\). Since \(\sqrt{5} > 2\), we must have that \(\frac{1}{\sqrt{5}} < \frac{1}{2}\) and so is closer to zero than
it is to one. Thus the claim holds for the case \(n=0\). Similarly,
for \(n=1\) we have \(\frac{\phi^1}{\sqrt{5}} \approx 0.7236\)
which is closer to one than it is to zero as needed. For the
inductive step, we assume that \(\text{Fib}(k)\) is the closest
integer to \(\frac{\phi^k}{\sqrt{5}}\) and show that \(\text{Fib}(k+1)\)
is the closest integer to \(\frac{\phi^{k+1}}{\sqrt{5}}\).
\end{document}
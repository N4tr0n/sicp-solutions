% Created 2019-01-31 Thu 12:33
% Intended LaTeX compiler: pdflatex
\documentclass[11pt]{article}
\usepackage[utf8]{inputenc}
\usepackage[T1]{fontenc}
\usepackage{graphicx}
\usepackage{grffile}
\usepackage{longtable}
\usepackage{wrapfig}
\usepackage{rotating}
\usepackage[normalem]{ulem}
\usepackage{amsmath}
\usepackage{textcomp}
\usepackage{amssymb}
\usepackage{capt-of}
\usepackage{hyperref}
\author{Nathan Hitchings}
\date{\today}
\title{Assignment 6 Solutions}
\hypersetup{
 pdfauthor={Nathan Hitchings},
 pdftitle={Assignment 6 Solutions},
 pdfkeywords={},
 pdfsubject={},
 pdfcreator={Emacs 26.1 (Org mode 9.2)},
 pdflang={English}}
\begin{document}

\maketitle
\section{Exercise 1.21}
\label{sec:org5c12f83}
Use the \texttt{smallest-divisor} procedure to find the smallest divisor of each of
the following numbers: 199, 1999, 19999.
\subsection{Solution}
\label{sec:orga43f445}
This is the \texttt{smallest-divisor} procedure:
\begin{verbatim}
#lang sicp

(define (smallest-divisor n) (find-divisor n 2))

(define (find-divisor n test-divisor)
  (cond ((> (square test-divisor) n) n)
        ((divides? test-divisor n) test-divisor)
        (else (find-divisor n (+ test-divisor 1)))))

(define (divides? a b) (= (remainder b a) 0))

(define (square x) (* x x))
\end{verbatim}
Applying \texttt{smallest-divisor} to each of 199, 1999, and 19999 yields
\begin{verbatim}
(smallest-divisor 199)   ; = 199
(smallest-divisor 1999)  ; = 1999
(smallest-divisor 19999) ; = 7
\end{verbatim}
\section{Exercise 1.22}
\label{sec:org1720b7b}
Most Lisp implementations include a primitive called \texttt{runtime} that returns an
integer that specifies the amount of time the system has been running
(measured, for example, in microseconds). The following \texttt{timed-prime-test}
procedure, when called with an integer \emph{n}, prints \emph{n} and checks to see if
\emph{n} is prime. If \emph{n} is prime, the procedure prints three asterisks followed
by the amount of time used in performing the test.
\begin{verbatim}
#lang sicp

(define (prime? n)
  (= n (smallest-divisor n)))

(define (timed-prime-test n)
  (newline)
  (display n)
  (start-prime-test n (runtime)))

(define (start-prime-test n start-time)
  (if (prime? n)
      (report-prime (- (runtime) start-time))))

(define (report-prime n elapsed-time)
  (display " *** ")
  (display elapsed-time))
\end{verbatim}
Using this procedure, write a procedure \texttt{search-for-primes} that checks the
primality of consecutive odd integers in a specified range. Use your procedure
to find the three smallest primes larger than 1000; larger than 10,000; larger
than 100,000; larger than 1,000,000. Note the time needed to test each
prime. Since the testing algorithm has order of growth of \(\theta(\sqrt{n})\), you should expect that testing for primes around 10,000 should take about
\(\sqrt{10}\) times as long as testing for primes around 1000. Do your
timing data bear this out? How well do the data for 100,000 and 1,000,000
support the \(\theta(\sqrt{n})\) prediction? Is your result compatible with
the notion that programs on your machine run in time proportional to the
number of steps required for the computation?
\subsection{Solution}
\label{sec:orgae83045}
\begin{verbatim}
(define (search-for-primes start end)
  (cond ((> start end) (display "Done\n"))
        ((even? start) (search-for-primes (+ start 1) end))
        (else (begin
                (timed-prime-test startp)
                (search-for-primes (+ start 1) end)))))
\end{verbatim}
\begin{center}
\begin{tabular}{rr}
prime number & time (in microseconds?)\\
\hline
1009 & 1\\
1013 & 1\\
1019 & 1\\
10007 & 2\\
10009 & 1\\
10037 & 1\\
100003 & 5\\
100019 & 5\\
100043 & 5\\
1000003 & 15\\
1000033 & 15\\
1000037 & 15\\
10000019 & 48\\
10000079 & 48\\
10000103 & 47\\
\end{tabular}
\end{center}
\section{Exercise 1.23}
\label{sec:org677a77d}
The \texttt{smallest-divisor} procedure shown at the start of this section does lots
of needless testing: After it checks to see if the number is divisible by 2
there is no point in checking to see if it is divisible by any larger even
numbers. This suggests that the values used for test-divisor should not be 2,
3, 4, 5, 6, \ldots{}, but rather 2, 3, 5, 7, 9, \ldots{}. To implement this change,
define a procedure \texttt{next} that returns 3 if its input is equal to 2 and
otherwise returns its input plus 2. Modify the \texttt{smallest-divisor} procedure to
use \texttt{(next test-divisor)} instead of \texttt{(+ test-divisor 1)}. With
\texttt{timed-prime-test} incorporating this modified version of \texttt{smallest-divisor},
run the test for each of the 12 primes found in Exercise 1.22. Since this
modification halves the number of test steps, you should expect it to run
about twice as fast. Is this expectation confirmed? If not, what is the
observed ratio of the speeds of the two algorithms, and how do you explain the
fact that it is different from 2?
\subsection{Solution}
\label{sec:org4441bec}
\begin{verbatim}
(define (smallest-divisor n)
  (find-divisor n 2))

(define (find-divisor n test-divisor)
  (define (next x)
    (cond [(= 2 x) 3]
          [else (+ x 2)]))
  (cond ((> (square test-divisor) n) n)
        ((divides? test-divisor n) test-divisor)
        (else (find-divisor n (next test-divisor)))))
\end{verbatim}
This is the output from running this version of the program on the primes in
thne previous exercise:
\begin{center}
\begin{tabular}{rr}
prime & time\\
\hline
1009 & 5\\
1013 & 1\\
1019 & 1\\
10007 & 4\\
10009 & 4\\
10037 & 4\\
100003 & 6\\
100019 & 23\\
100043 & 6\\
1000003 & 16\\
1000033 & 16\\
1000037 & 16\\
10000019 & 61\\
10000079 & 50\\
10000103 & 50\\
\end{tabular}
\end{center}
\section{Exercise 1.24}
\label{sec:orgccbd820}
Modify the \texttt{timed-prime-test} procedure of Exercise 1.22 to use \texttt{fast-prime?}
(the Fermat method), and test each of the 12 primes you found in that
exercise. Since the Fermat test has \(\theta(\log n)\) growth, how would you
expect the time to test primes near 1,000,000 to compare with the time needed
to test primes near 1000? Do your data bear this out? Can you explain any
discrepancy you find?
\subsection{Solution}
\label{sec:org4124d2f}
\begin{center}
\begin{tabular}{rr}
prime & time\\
\hline
1009 & 276\\
1013 & 4\\
1019 & 5\\
10007 & 5\\
10009 & 5\\
10037 & 5\\
100003 & 16\\
100019 & 6\\
100043 & 5\\
1000003 & 6\\
1000033 & 6\\
1000037 & 6\\
10000019 & 7\\
10000079 & 8\\
10000103 & 7\\
\end{tabular}
\end{center}
\section{Exercise 1.25}
\label{sec:orga53046e}
Alyssa P. Hacker complains that we went to a lot of extra work in writing
\texttt{expmod}. After all, she says, since we already know how to compute
exponentials, we could have simply written
\begin{verbatim}
(define (expmod base exp m)
  (remainder (fast-expt base exp) m))
\end{verbatim}
Is she correct? Would this procedure serve as well for our fast prime tester?
Explain.
\section{Exercise 1.26}
\label{sec:orgec7b09d}
Louis Reasoner is having great difficulty doing Exercise 1.24. His
\texttt{fast-prime?} test seems to run more slowly than his \texttt{prime?} test. Louis
calls his friend Eva Lu Ator over to help. When they examine Louis’s code,
they find that he has rewritten the \texttt{expmod} procedure to use an explicit
multiplication, rather than calling \texttt{square}:
\begin{verbatim}
(define (expmod base exp m)
  (cond ((= exp 0) 1)
        ((even? exp)
         (remainder (* (expmod base (/ exp 2) m)
                       (expmod base (/ exp 2) m))
                    m))
        (else
         (remainder (* base
                       (expmod base (- exp 1) m))
                    m))))
\end{verbatim}
``I don't see what difference that could make,'' says Louis. ``I do.'' says
Eva. ``By writing the procedure like that, you have transformed the \(\theta(\log n)\) process into a \(\theta(n)\) process.'' Explain.
\subsection{Solution}
\label{sec:orgfe7ed83}
The reason that the process is now \(\theta(n)\) is that the expression
\texttt{(expmod base (/ exp 2) m)} must now be evaluated twice for every call while
using \texttt{square} allows for the expression to be evaluted only once.
\section{Exercise 1.27}
\label{sec:orge743f88}
Demonstrate that the Carmichael numbers listed in Footnote 1.47 really do fool
the Fermat test. That is, write a procedure that takes an integer \emph{n} and
tests whether \(a^n\) is congruent to \emph{a} modulo \emph{n} for every \(a < n\),
and try your procedure on the given Carmichael numbers.
\subsection{Solution}
\label{sec:org90d2632}
\begin{verbatim}
(define (square x) (* x x))
(define (expmod base exp m)
  (cond [(= exp 0) 1]
        [(even? exp)
         (remainder (square (expmod base (/ exp 2) m))
                    m)]
        [else
         (remainder (* base (expmod base (- exp 1) m))
                    m)]))
(define (test n)
  (define (try-it a)
    (= (expmod a n n) a))
  (define (loop a)
    (cond [(= a 0) true]
          [(try-it a) (loop (- a 1))]
          [else false]))
  (loop n))
\end{verbatim}
\section{Exercise 1.28}
\label{sec:org2b25a3a}
One variant of the Fermat test that cannot be fooled is called the
\emph{Miller-Rabin} test (Miller 1976; Rabin 1980). This starts from an alternate
form of Fermat’s Little Theorem, which states that if \emph{n} is a prime number
and \emph{a} is any positive integer less than \emph{n}, then \emph{a} raised to the \((n-1)\)-st power is congruent to 1 modulo \emph{n}. To test the primality of a number
\emph{n} by the Miller-Rabin test, we pick a random number \(a<n\) and raise \emph{a}
to the \((n-1)\)-st power modulo \emph{n} using the \texttt{expmod} procedure. However,
whenever we perform the squaring step in \texttt{expmod}, we check to see if we have
discovered a ``nontrivial square root of 1 modulo \emph{n},'' that is, a number not
equal to 1 or \(n-1\) whose square is equal to 1 modulo \emph{n}. It is possible
to prove that if such a nontrivial square root of 1 exists, then \emph{n} is not
prime. It is also possible to prove that if \emph{n} is an odd number that is not
prime, then, for at least half the numbers \(a<n\), computing \(a^{n-1}\)
in this way will reveal a nontrivial square root of 1 modulo \emph{n}. (This is why
the Miller-Rabin test cannot be fooled.)  Modify the \texttt{expmod} procedure to
signal if it discovers a nontrivial square root of 1, and use this to
implement the Miller-Rabin test with a procedure analogous to
\texttt{fermat-test}. Check your procedure by testing various known primes and
non-primes. Hint: One convenient way to make \texttt{expmod} signal is to have it
return 0.
\subsection{Solution}
\label{sec:org47d7e61}
\begin{verbatim}
(define (square x) (* x x))
(define (expmod base exp m)
  (cond [(= exp 0) 1]
        [(even? exp)
         (remainder (square (expmod base (/ exp 2) m))
                    m)]
        [else
         (remainder (* base (expmod base (- exp 1) m))
                    m)]))
\end{verbatim}
\end{document}